%%%%%%%%%%%%%%%%%%%%%%%%%%%%%%%%%%%%%%%%%%%%%%%%%%%%%%%%%%%%%%%%%%%%%%%%
%% Examples of how to use shortex.sty
%%%%%%%%%%%%%%%%%%%%%%%%%%%%%%%%%%%%%%%%%%%%%%%%%%%%%%%%%%%%%%%%%%%%%%%%


\documentclass{article}
\usepackage[autonum,colorhypersetup]{shortex}

\title{Examples of how to use \texttt{shortex.sty}}
\author{Created by Jonathan Huggins and Trevor Campbell}
\date{Updated \today}


\begin{document}

\maketitle

\section{Brackets and bracket-like functions}

You can specify a bracket size using $-1,\dots,4$, where $-1$ uses \texttt{\textbackslash left} and \texttt{\textbackslash right}, $0$ uses nothing, and positive numbers 
use increasingly large fixed sizes. When inline, bracket sizes are not adjusted.

\bitems
\item Regular brackets:  \verb!\rbra{\frac{x}{y}}!
    \bitems
    \item Inline: $\rbra{\frac{x}{y}}$ 
    \item Display: $\displaystyle\rbra{\frac{x}{y}}$
    \eitems
\item Curly brackets: \verb!\cbra[2]{\frac{x}{y}}!
    \bitems
    \item Inline: $\cbra[2]{\frac{x}{y}}$
    \item Display: $\displaystyle\cbra[2]{\frac{x}{y}}$
    \eitems
\item Square brackets: \verb!\sbra[4]{\frac{x}{y}}!
    \bitems
    \item Inline: $\sbra[4]{\frac{x}{y}}$
    \item Display: $\displaystyle\sbra[4]{\frac{x}{y}}$
    \eitems
\eitems
Other bracket-like, semantic command are also available, including \verb!\abs!, \verb!\set!, \verb!\floor!, \verb!\ceil!, \verb!\norm!, \verb!\inner!, and \verb!\card!. 


\section{annotation commands}
\begin{tabular}{ll}
    \verb!\barA! & $\barA$ \\
    \verb!\bara! & $\bara$ \\
    \verb!\bA! & $\bA$ \\
    \verb!\bB! & $\bB$ \\
    \verb!\balpha! & $\balpha$ \\
    \verb!\bGamma! & $\bGamma$ \\
    \verb!\mcA! & $\mcA$ \\
    \verb!\hmcA! & $\hmcA$ \\
    \verb!\mfA! & $\mfA$ \\
    \verb!\mfa! & $\mfa$ \\
    \verb!\bmfA! & $\bmfA$ \\
    \verb!\bmfa! & $\bmfa$ \\
    \verb!\hA! & $\hA$ \\
    \verb!\ha! & $\ha$ \\
    \verb!\halpha! & $\halpha$ \\
    \verb!\hGamma! & $\hGamma$ \\
    \verb!\bhA! & $\bhA$ \\
    \verb!\bha! & $\bha$ \\
    \verb!\bhalpha! & $\bhalpha$ \\
    \verb!\bhGamma! & $\bhGamma$ \\
    \verb!\whA! & $\whA$ \\
    \verb!\wha! & $\wha$ \\
    \verb!\tdA! & $\tdA$ \\
    \verb!\tda! & $\tda$ \\
    \verb!\tdalpha! & $\tdalpha$ \\
    \verb!\tdGamma! & $\tdGamma$ \\
    \verb!\btdA! & $\btdA$ \\
    \verb!\btda! & $\btda$ \\
    \verb!\btdalpha! & $\btdalpha$ \\
    \verb!\btdGamma! & $\btdGamma$ \\
    \verb!\biA! & $\biA$ \\
    \verb!\bia! & $\bia$ \\
    \verb!\bhiA! & $\bhiA$ \\
    \verb!\bhia! & $\bhia$ \\
\end{tabular}


\end{document}



